\documentclass[letterpaper,11pt]{article}
\usepackage{fullpage}
\usepackage{amsfonts}
\usepackage{amsmath}
\usepackage{amssymb}
\usepackage{multirow}
\usepackage{hyperref}
\usepackage{fancyvrb,alltt}

\DefineVerbatimEnvironment{ZKPVerbatim}{Verbatim}
	{fontfamily=tt,tabsize=4,numbers=left,numbersep=2pt,xleftmargin=2pt,
	 fontsize=\footnotesize,frame=single,commandchars=\\\{\}}

\hbadness=10000

\newcommand{\tbf}[1]{\textbf{#1}}

%opening
\title{Description of ZKPDL}
%\author{Brownies}

\begin{document}

\maketitle


\section{Syntax}\label{sec:syntax}

A ZKPDL program consists of two blocks: a \verb#computation# block and a
\verb#proof# block.
Both blocks are optional; if a user wants a calculator (for either modular or
integer arithmetic) then he can use just the \verb#computation# block, but if he 
has all
his values pre-computed and wants the zero-knowledge proof, he can use just
the \verb#proof# block.

\subsection{The computation block}\label{sec:computation}

The \verb#computation# block is itself comprised of two blocks: the 
\verb#given# blockand the \verb#compute# block.  A sample computation block is
as follows:

\begin{ZKPVerbatim}
computation:
	given: 
		group: G = <g,h>
		group: G2
		group: G3 = <f1,f2>
			modulus: N
		elements in G: a, b, c
		exponents in G: x_1, x_2
		integers: y, z
	compute:
		random exponents in G: r_1, r_2
		random integer in [0,N): s
		random prime of length 2^(y+z): p
		c_1 := g^x_1 * h^r_1
\end{ZKPVerbatim}

\subsubsection{The given block}\label{sec:given}

This block names all the values necessary for the computation.  The sample
program above demonstrates the four types of values that can be declared here: 
group objects, group elements, group exponents, and integers.  Group names
must start with a letter and cannot have any subscripts; sample group
declarations are in lines $3$, $4$, and $5$ of the sample program.  The sample
program also demonstrates the various options the programmer has when
declaring a group.  If the generators for the group are to be used later on as
bases (which we can see being done here in line $14$), then the generators can
be named.  Furthermore, if the modulus of the group is to be used, it can also
be named as seen on line $6$.  

The declaration of numerical objects (group elements, group exponents, or
integers) are relatively straightforward; the only small note is that for both 
group elements and group exponents, a group name must be specified as seen
in lines $7$ and $8$.  

Every value declared in the \verb#given# block must be input to the
interpreter at compute time, and an exception will be thrown if any are
missing.

\subsubsection{The compute block}\label{sec:compute}

As seen in our sample program, there are two main types of computations that
can be carried out: picking a random value with some given set of constraints,
or performing an arithmetic operation.  For the first type, there are three
possible operations currently supported, as seen in lines $11$, $12$, and
$13$; note that in each of these, the value on the right side of the colon
will be assigned as it is being declared.  On line $11$, random 
exponents in a given group are picked according to the order of the group.
On line $12$, a random number is being chosen in a given range; note that this
range can contain arithmetic expressions (for example $2^x$) and is not limited
to integers or variable names.  Similarly, on line $13$ we see a prime number
being picked of a certain length; again, the length can be given as an
arbitrary arithmetic expression.

Finally, on line $14$, we see a basic arithmetic operation being carried out.
If this operation is carried out over the integers, the type for the variable
being assigned will be an integer.  If it is carried out in a group (as is the
case here) then the types of the values will be checked to make sure they
match up.  For example, if the line were changed to read 
\verb#c_1 = g^x_1 * f1^r_1#, then a compile-time exception would be thrown, as 
elements from
different groups cannot be multiplied together.  Similarly, if the line read
\verb#c_1 = x_1^g * h^r_1# an exception would again be thrown, this time
because the user is attempting to raise a group exponent to a group element
power, which is also prohibited.  As it is written, however, line $14$ is
allowed and the resulting \verb#c_1# value will be typed as an element in the
group \verb#G# (as well as assigned whatever value the right-hand expression
evaluates to).

\subsection{The proof block}\label{sec:proof}

The proof block consists of three blocks: the \verb#given# block, the 
\verb#prove knowledge of# block, and the \verb#such that# block.  Here is a
sample proof block:

\begin{ZKPVerbatim}
proof:
	given: 
		group: G = <g,h>
		integers: lower, upper
		element in G: c1, c2, c3, d, f
			commitment to x1: c1 = g^x1 * h^r1
			commitment to x2: c2 = g^x2 * h^r2
			commitment to x3: c3 = g^x3 * h^r3
			commitment to y: d = g^y * h^r
	prove knowledge of:
		exponents in G: x1, x2, x3, y, r1, r2, r3, r
	such that:
		f = c1^(x2+x3) * c2^r
		x1 = x2 * x3
		x3 = y^2
		range: lower <= y < upper
\end{ZKPVerbatim}  

The first block, the \verb#given# 
block, is almost syntactically identical to the \verb#given# block described in
Section~\ref{sec:given}.  The significance of the block has changed though;
whereas before it specified the values needed to perform the computations, it now
specifies the common values shared between the prover and verifier.  With this
in mind, we can look at lines $6$ through $9$ to see the major difference in
the \verb#given# block, which is that commitment forms can be specified.  In
this case, each of the commitments is a commitment to a single value, but
commitments can also be formed for multiple values (in which case all the
values are listed in the commitment line).

The \verb#prove knowledge of# block is syntactically fairly simple.  It lists,
separated by type, the private values that the prover knows and the verifier
doesn't.  This block may contain elements, exponents, and integers, but not
groups (as groups are always regarded as common parameters).

Finally, the \verb#such that# block specifies the relations between the values
for which the prover is creating a zero-knowledge proof.  As seen in lines
$13$ through $16$, there are four types of proofs that can be specified.  The
first, on line $13$, is proving knowledge of the discrete logarithm form of a
value; in this case, proving knowledge of the values \verb#x2#, \verb#x3#, and
\verb#r# such that this equation is satisfied.  The next type of proof, on
line $14$, is proving that a secret value is the product of two other secret
values.  The next type, on line $15$, is just the special case of the
multiplication proof in which the two secret values are the same.  Finally, on
line $16$, we see range proofs, in which the prover proves that a hidden value
is contained within some public range.

\subsection{Syntactic sugar}

To make programs easier to write, we have added in a number of additional 
features.  We demonstrate these features in the following program:

\begin{ZKPVerbatim}
computation:
	given:
		group: G = <g,h[1:k]>
		exponents in G: x[1:k], v[1:k], x
	compute:
		random exponents in G: r[1:k]
		for(i, 1:k, c_i := g^x_i * h_i^r_i)
		C := g^x * for(i, 1:k, *, h_i^v_i)

proof:
	given:
		group: G = <g,h[1:k]>
		elements in G: c[1:k], C
			for(i, 1:k, commitment to x_i: c_i = g^x_i * h_i^r_i)
		integer: len
	prove knowledge of:
		exponents in G: x[1:k], v[1:k], x, r[1:k]
	such that:
		for(i, 1:k, range: (-(2^len - 1)) <= x_i < 2^len)
		C = g^x * for(i, 1:k, *, h_i^r_i)
\end{ZKPVerbatim}

As we can see in the program, there are two main features introduced: the use
of \verb#x[1:k]# as shorthand for \verb#x_1,...,x_k# and the use of for loops.
The first kind of for loop, which we see on lines $8$ and $20$, is similar to
$\sum$ or $\prod$ in mathematical notation.  After the \verb#for# symbol, the 
index used is named, followed by the values over which the index should range, 
followed by the operation (either addition or multiplication), followed by the
values which should either be summed or multiplied.  So, writing 
\verb#for(i, 1:3, +, x_i)# is just shorthand for writing 
\verb#x_1 + x_2 + x_3#.  

The other type of for loop is more similar to for loops in traditional
programming languages.  We can see it used in lines $7$, $14$, and $19$.

Finally, we also note the addition of the value \verb#k# in the above program.
This is very useful in allowing us to write programs for protocols which allow
a variable number of inputs (for example, forming a CL signature on $k$
values).  At compile time, the value for $k$ must be given to the interpreter.
The constant is then substituted in and propagated throughout the program (and
any subexpressions in which it is used are evaluated); more on this can be
found in Section~\ref{sec:usage}.

\section{Usage}\label{sec:usage}

Our interpreter, when given a ZKPDL program, works in two main stages.  In the
first, which we refer to as \emph{compile time}, the program is checked for 
syntactic correctness and a number of
optimizations are run.  First, any constants are substituted in and then
propagated, and any for loops are unrolled.  The interpreter then performs a
number of safety checks.  It first checks to see if any variables are being
used that were not previously declared, and also checks to see any variables
are being declared without being used.  It will stop and throw an exception 
if the former problem occurs, but in the latter case it will just
print out a warning message.  It will then check that group types match up,
and again throw an exception if this check fails.  It then performs a
number of operations to set up the environment for the next phase, and ends by 
caching a copy of itself so that if the
same program is run again this phase can be skipped.  No values need to be
given to the interpreter at compile time, although if groups are given for
which the generators will be used as bases for
multi-exponentiations then the interpreter will perform caching to
speed up the exponentation when it is performed later on.

In the second phase, \emph{compute time}, the interpreter is given all the
values needed for the computation (although as mentioned above, it may have
already been given the groups).  If it is acting as the prover, it performs 
all the computations
specified in the \verb#computation# block, followed by all the operations
needed to output the zero-knowledge proof specified in the \verb#proof# block.
If it is the verifier, it will also take in the proof, and then perform all
the computations necessary to verify the proof's validity.

Here is a sample usage, in C++, of our interpreter:

\begin{alltt}
\footnotesize{

hashalg_t hashAlg = Hash::SHA1;
int stat = 80;

const GroupPrime* grp = parameters->getCashGroup();
ZZ x1 = 2;
ZZ x2 = 3;
group_map pg;
variable_map pv;
pg["G"] = grp;
pv["x_1"] = x1;
pv["x_2"] = x2;

InterpreterProver prover;
prover.check("multiplication.txt", pg);
prover.compute(pv);
ProofMessage proofMsg = ProofMessage(prover.getPublicVariables(),
					 		 		 prover.computeProof(hashAlg));

group_map vg;
variable_map vv;
vg["G"] = parameters->getCashGroup();

InterpreterVerifier verifier;
verifier.check("multiplication.txt", vg);
verifier.compute(vv, proofMsg.publics);
bool verified = verifier.verify(proofMsg.proof);
}
\end{alltt}

\section{ZKPDL Grammar}\label{sec:grammar}

%\begin{figure}[h]
To enhance the overview of our language, we provide here a full EBNF
  grammar.\\

 \fbox{
\begin{minipage}[c]{6 in}

 \centering

\raggedright
\tt
\footnotesize
%\small
\noindent\tbf{spec} := (computation)? (proof)?\\
\tbf{computation} := ``computation'' COLON ``given'' COLON 
givenList ``compute'' COLON computeRandomList 
computeEquationList\\
\tbf{proof} := ``proof'' COLON ``given'' COLON givenList 
			   ``prove'' ``knowledge'' ``of'' COLON knowledgeList 
			   ``such'' ``that'' COLON suchThatList\\
\tbf{suchThatList} := (suchThatRel)+\\
\tbf{suchThatRel} := equalRelation \vline~rangeRelation \vline~forRelation\\
\tbf{knowledgeList} := (exponentDecl \vline~integerDecl)+\\
\tbf{givenList} := (groupDecl \vline~elementsDecl \vline~exponentDecl \vline~integerDecl)+\\
\tbf{randomBndDecl} := (``integer'' \vline~``integers'') ``in'' LBRACKET expr COMMA expr
				 RPAREN COLON idGeneralDeclList\\
\tbf{randomPrimeDecl} := (``prime'' \vline~``primes'') ``of'' ``length'' expr COLON 
				   idGeneralDeclList\\
\tbf{computeRandomList} := (``random'' (randExponentDecl  \vline~randomPrimeDecl  \vline~randomBndDecl))*\\
\tbf{groupDecl} := ``group'' COLON identifierDecl (EQUAL setDecl)? (``modulus'' COLON
			 subscriptIdentifierDecl)?\\
\tbf{randExponentDecl} := (``exponent'' \vline~``exponents'') ``in'' identifier COLON 
					idGeneralDeclList\\
\tbf{exponentDecl} := (``exponent'' \vline~``exponents'') ``in'' identifier COLON
				idGeneralDeclList\\
\tbf{elementsDecl} := (``element'' \vline~``elements'') ``in'' identifier COLON 
				idGeneralDeclList (elementsEquationList)?\\
\tbf{integerDecl} := (``integer'' \vline~``integers'') COLON idGeneralDeclList\\
\tbf{computeEquationList} := (computeEquation)+\\
\tbf{computeEquation} := equalDeclRelation  \vline~comRelation  \vline~forRelation\\
\tbf{forRelation} := ``for'' LPAREN identifier COMMA expr COLON expr COMMA
			   (rangeRelation  \vline~genEqual) RPAREN\\
\tbf{forCom} := ``for'' LPAREN identifier COMMA expr COLON expr COMMA
		  comRelation RPAREN\\
\tbf{elementsEquationList} := (elementsEquation)+\\
\tbf{elementsEquation} := comRelation  \vline~forCom\\
\tbf{comRelation} := ``commitment'' ``to'' idSubList COLON subscriptIdentifier
			   EQUAL expr\\
\tbf{genEqual} := identifier (SUBSCRIPT (ID  \vline~INTLIT))? (EQUAL \vline~CEQUAL) expr\\
\tbf{equalRelation} := subscriptIdentifier EQUAL expr\\
\tbf{equalDeclRelation} := subscriptIdentifierDecl CEQUAL expr\\
\tbf{rangeRelation} := ``range'' COLON expr (LTHAN \vline~LEQ) expr
(LTHAN \vline~LEQ) expr  \vline~(GTHAN \vline~GEQ) expr (GTHAN \vline~GEQ) expr\\
\tbf{expr} := prodExpr (ADD prodExpr  \vline~SUB prodExpr)*\\
\tbf{forExpr} := ``for'' LPAREN identifier COMMA expr COLON expr COMMA
		   (ADD  \vline~MUL) COMMA expr RPAREN\\
\tbf{prodExpr} := powExpr (MUL powExpr  \vline~DIV powExpr)*\\
\tbf{powExpr} := unaryExpr  \vline~baseExpr\\
\tbf{unaryExpr} := SUB baseExpr  \vline~baseExpr\\
\tbf{baseExpr} := INTLIT  \vline~subscriptIdentifier  \vline~LPAREN expr RPAREN  \vline~forExpr\\
\tbf{setDecl} := LTHAN idGeneralDeclList GTHAN\\
\tbf{idSubDeclList} := (subscriptIdentifierDecl)+\\
\tbf{idGeneralDeclList} := (idDeclGeneral)+\\
\tbf{idDeclList} := (identifierDecl)+\\
\tbf{idSubList} := (subscriptIdentifier)+\\
\tbf{idList} := (identifier)+\\
\tbf{idDeclGeneral} := idDeclRange  \vline~subscriptIdentifierDecl\\
\tbf{idDeclRange} := identifierDecl LBRACKET expr COLON expr RBRACKET\\
\tbf{subscriptIdentifierDecl} := identifier (SUBSCRIPT (ID  \vline~INTLIT))?\\
\tbf{identifierDecl} := ID\\
\tbf{subscriptIdentifier} := identifier (SUBSCRIPT (ID  \vline~INTLIT))?\\
\tbf{identifier} := ID\\

% \noindent\tbf{LPAREN} := ``(''\\
% \tbf{RPAREN} := ``)''\\
% \tbf{LBRACKET} := ``[''\\
% \tbf{RBRACKET} := ``]''\\
% \tbf{LCURLY} := ``{''\\
% \tbf{RCURLY} := ``}''\\
% \tbf{COMMA} := ``,''\\
% \tbf{SCOLON} := ``;''\\
% \tbf{COLON} := ``:''\\
% \tbf{CEQUAL} := ``:'' ``=''\\
% \tbf{EQUAL} := ``=''\\
% \tbf{LTHAN} := ``<''\\
% \tbf{GTHAN} := ``>''\\
% \tbf{LEQ} := ``<'' ``=''\\
% \tbf{GEQ} := ``>'' ``=''\\
% \tbf{ADD} := ``+''\\
% \tbf{SUB} := ``-''\\
% \tbf{MUL} := ``*''\\
% \tbf{DIV} := ``/''\\
% \tbf{POW} := ``\^''\\
% \tbf{SUBSCRIPT} := ``\_''\\
\tbf{ID} := ('a'..'z'  \vline~'A'..'Z')('a'..'z' \vline~'A'..'Z' \vline~'0'..'9')*\\
\tbf{INTLIT} := '0'  \vline~('1'..'9')('0'..'9')*\\

\end{minipage}
}
%\label{fig:ebnf}
%\end{figure}



\end{document}

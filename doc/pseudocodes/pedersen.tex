\subsection{Pedersen Commitment Scheme}

\textsc{Overview}:\\
The Pedersen commitment scheme \cite{pedersen:secret-sharing:c91} is a statistically hiding, computationally binding commitment scheme. It allows for commitments to values between $1$ and $ \primeOrder-1 $.


\textsc{Setup}:\\
This commitment scheme uses a prime-order group.  If an untrusted party (\eg the Verifier) generates the prime-order group, then the participants (both the Committer and the Verifier) need to check that both $ \primeMod $ and $ \primeOrder $ are primes (such that $ \primeOrder $ divides $ \primeMod - 1 $) and that $g_i,h$ have order $ \primeOrder $ (which is equivalent to saying that $g_i \neq 1 \mod \primeMod$ and $g_{i}^{\primeOrder} = 1 \mod \primeMod$). It is important that the Committer does not know the relative discrete logarithms of the bases, or otherwise the commitment is no longer binding.


\textsc{Assumptions}:\\
The security of the scheme relies on the Discrete Logarithm assumption.


\textsc{Hiding}:\\
The hiding property relies on the fact that $g_i,h$ all generate the group with prime order $ \primeOrder $, so that when randomization is used the resulting commitment is a random element of the prime-order group.


\textsc{Binding}:\\
The binding property relies on the Committer not being able to find two different openings, using the same bases, that result in the same commitment.  This follows directly from the Discrete Logarithm assumption.



\begin{algorithm}[H]\label{commitP}
\SetKwComment{Comment}{}{}
\SetKwInput{Pre}{Pre-conditions}
\SetKwInput{Post}{Post-conditions}
\dontprintsemicolon

\KwIn{Definition of the prime-order group, number of secrets $k$, bases $g_1,\ldots,g_k,h$, secrets $x_1,\ldots,x_k$}
\Pre{Group and the bases must adhere to guidelines in the \textsc{Setup} in this section. Each $x_i$ must be between $1$ and $ \primeOrder-1 $.}
\KwOut{commitment $C$, opening $open$}
\Post{$C$ should be sent to the Verifier.}
\BlankLine

\Comment{Commit} \;
\Indp
  Pick a random number $r$ from $Z_{\primeOrder}^*$. \;
  Create the commitment $C = h^r \prod_{i=1}^{k} g_i^{x_i} \mod \primeMod $. \;
  Output commitment $C$ and opening $open = x_1,\ldots,x_k,r$. \;
\Indm

\caption{Commitment procedure of the Pedersen commitment scheme. This procedure is run by the Committer.}
\end{algorithm}


To open a commitment, the Committer simply sends $open$ to the Verifier. Upon receiving the opening, the Verifier needs to verify its validity. This is done as follows:


\begin{algorithm}[H]\label{verifyP}
\SetKwComment{Comment}{}{}
\SetKwInput{Pre}{Pre-conditions}
\SetKwInput{Post}{Post-conditions}
\dontprintsemicolon

\KwIn{Definition of the prime-order group, number of secrets $k$, bases $g_1,\ldots,g_k,h$, commitment $C$, opening $open = x_1,\ldots,x_k,r$}
\Pre{The group may not be generated by the Committer. Each $x_i,r$ must be between $1$ and $ \primeOrder-1 $.}
\KwOut{$\accept$ or $\reject$}
%\Post{}
\BlankLine

\Comment{Verify} \;
\Indp
  \If{$C = h^r \prod_{i=1}^{k} g_i^{x_i} \mod \primeMod $}
    {Output $\accept$ \;}
  \Else
    {Output $\reject$ \;}
\Indm

\caption{Verification procedure of the Pedersen commitment scheme. This procedure is run by the Verifier after receiving the opening $open$ for a commitment $C$.}
\end{algorithm}



\textsc{Comments}:\\
If only one generator exists (namely $g_1$), then the commitment is only computationally hiding based on the Discrete Logarithm assumption, although it still works.
